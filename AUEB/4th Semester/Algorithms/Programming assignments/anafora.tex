\documentclass[12pt]{article}

\usepackage[utf8]{inputenc}
\usepackage[greek, english]{babel}
\usepackage{alphabeta}

\title{3η Σειρά προγραμματιστικών ασκήσεων}
\author{Δημήτριος Τσιομπίκας 3180223}
\begin{document}

\maketitle

Άσκηση 3.1\\ \\
Στην άσκηση αυτή υλοποίησα μία mergesort στην οποία μετριέται το πλήθος των αντιστροφών.\\

Άσκηση 3.2 \\ \\
Στην άσκηση αυτή υλοποίησα το πρόγραμμα ως εξής : βρίσκω τις συχνότητες των γραμμάτων στο text file , κατασκευάζω το δέντρο huffman και εμφανίζω την κωδικοποίηση του κάθε γράμματος στην οθόνη. \\

Άσκηση 3.3 \\ \\
Εδώ χρησιμοποίησα τον αλγόριθμο αθροίσματος υποσυνόλων , βάζω τους αριθμούς του text file σε έναν πίνακα και ελέγχω για όλα τα υποσύνολά του αν κάποιο έχει άθροισμα που ισούται με C. Επιστρέφω true ή false ανάλογα με το αποτέλεσμα. \\

Άσκηση 3.4 \\ \\
Εδώ διαβάζω τα στοιχεία από το αρχείο με ένα Linked Hash map ώστε να τα παίρνει με τη σειρά που τα διαβάζει. Ύστερα βρίσκω τα ζεύγη κόμβων και βρίσκω ποιο είναι το μεγαλύτερο από αυτά που υποδηλώνει το longest shortest path. Επιστρέφω αυτό και τη διάμετρο του γράφου. \\ \\

Άσκηση 3.5 \\ \\
Στην άσκηση αυτή κατάφερα να υλοποιήσω μόνο τον εξαντλητικό αλγόριθμο.Έχω χρησιμοποιήσει και τα 2 datasets και κάνω το εξής : βρίσκω όλα τα δίκτυα μέσω Linked Hash map με κλειδί String και value List$<$String$>$ (αντίστοιχα Integer,List$<$Integer$>$ για το άλλο dataset) , ελέγχω ποια από αυτά είναι κλίκες με την εξαντλητική αναζήτηση , δηλαδή παίρνω ένα υποσύνολο , το ελέγχω με όλα τα άλλα και μετά παίρνω το επόμενο , κάνω το ίδιο κ.ο.κ. Το σκεπτικό είναι ότι εφόσον έχουμε μη-κατευθυνόμενο γράφο , αν 2 κόμβοι έχουν το ίδιο στοιχείο στις λίστες τους τότε δημιουργείται κλίκα.Αφού βρω τις κλίκες βρίσκω ποια ή ποιες έχουν το μέγιστο πλήθος κόμβων (άρα είναι και μέγιστη κλίκα) και τις επιστρέφω. \\

Οδηγίες : για compile αρκεί να τρέξετε το exercises set3 \\
javac ExercisesSet3.java \\
και για run :  java ExercisesSet3 Exercise1.txt Exercise2.txt Exercise3.txt Exercise4.txt lesmiserables.txt  zachary\_karate\_club.txt \\
όπου lesmiserables το πρώτο dataset και zachary\_karate\_club το 2ο της άσκησης 3.5.

\end{document}